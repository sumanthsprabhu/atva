\documentclass{llncs}

\begin{document}

\pagestyle{empty}

\mainmatter

\title{VAAMANA: Underapproximation Refinement Based Concurrent Program Verification Tool}

%TODO: change the authors
\author{Alfred Hofmann\inst{1}
\and Ingrid Beyer\inst{1} \and
Anna Kramer\inst{1} \and Erika Siebert-Cole\inst{1} \and\\
Angelika Bernauer-Budiman\inst{2} \and
Martina Wiese\inst{2} \and Anita B\"urk\inst{3}}

%TODO: change
\authorrunning{Alfred Hofmann et al.}

%TODO: change
\institute{Springer-Verlag, Computer Science Editorial III,
Postfach 10 52 80,\\
69042 Heidelberg, Germany\\
\email{\{Hofmann, Beyer, Kramer, Erika.Siebert-Cole, LNCS\}@Springer.de}\\
\texttt{http://www.springer.de/comp/lncs/index.html}
\and
Springer-Verlag, Computer Science Production, Postfach 10 52 80,\\
69042 Heidelberg, Germany\\
\email{\{Bernauer, Wiese\}@Springer.de}
\and
Springer-Verlag, Marketing Management, Postfach 10 52 80,\\
69042 Heidelberg, Germany\\
\email{Buerk@Springer.de}}

\maketitle

\begin{abstract}
%should be at least 70 words and at most 150
We have developed a tool VAAMANA to verify safety properties of concurrent programs 
in bounded model checking setting. Given an ANSI-C or C++ shared memory 
concurrent program and safety properties as assert statements the tool 
underapproximates the dataflow of program using likely invariants. The 
likely invariants are generated by dynamic analysis and added as 
constraints in given program on which we perform bounded model checking. 
Further, the tool performs refinement of underapproximated dataflow using 
UNSAT proof. Our technique is applicable for different encoding techniques 
of concurrent program verification. We evaluate the performance of the tool 
on different encodings and benchmarks.
\end{abstract}

\section{Introduction}
%TODO: Add three to four sentences of current status of concurrent program verification tool
Sequentialization is a verification technique where concurrent program is transformed to nondeterministic 
sequential program by restricting dataflow. Context bounded analysis.

\subsection{Motivating Examples}

\section {Implementation}
%Overall design figure and explanation

\subsection {Likely Invariant Generation}
\subsection {Adding likely inavriants as constraints}
\subsection {Refinement}
\section {Experiments}

\section {Conclusion}

\begin{thebibliography}{4}
%
\bibitem{rama}
Ramalingam G. Context-sensitive synchronization-sensitive analysis is undecidable. 
ACM Transactions on Programming languages and Systems (TOPLAS). 2000 Mar 1;22(2):416-30.
%
\bibitem{kiss}
Qadeer S, Wu D. KISS: keep it simple and sequential. Acm sigplan notices. 2004 Jun 9;39(6):14-24.
%
\bibitem{lal}
Lal A, Reps T. Reducing concurrent analysis under a context bound to sequential analysis. 
Formal Methods in System Design. 2009 Aug 1;35(1):73-97.
%
\bibitem{tomasco}
Tomasco E, Inverso O, Fischer B, La Torre S, Parlato G. 
Verifying concurrent programs by memory unwinding. 
In International Conference on Tools and Algorithms for the Construction and Analysis of 
Systems 2015 Apr 11 (pp. 551-565). Springer Berlin Heidelberg.
%
\bibitem{anand}
Anand Yeolekar, Kumar Madhukar, Dipali Bhutada, Venkatesh R.
Sequentialization Using Timestamps.
%
\bibitem{alglave}
Alglave J, Kroening D, Tautschnig M. 
Partial orders for efficient bounded model checking of concurrent software. 
In International Conference on Computer Aided Verification 2013 Jul 13 (pp. 141-157). 
Springer Berlin Heidelberg.
%
\end{thebibliography}

\end{document}
